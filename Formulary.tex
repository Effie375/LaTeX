\documentclass[10pt,a4paper]{article}

% Set the font (output) encoding
\usepackage[LGR]{fontenc}

% Greek-specific commands
\usepackage[greek]{babel}

\author{Ευστρατία Καζή}
\title{Μαθηματική Ανάλυση}
\date{\today}

\begin{document}

\begin{titlepage}
\maketitle
\end{titlepage}

\tableofcontents
\pagebreak

\section{Θεωρία συνόλων}
\subsection{Πληθάριθμος}
Πληθικός Αριθμός ή Πληθάριθμος ενός συνόλου Α είναι ο αριθμός των στοιχείων που περιέχει. Συμβολισμός: $\mathbf{card}(A)$

\begin{itemize}
\item $A = \{a, b, c, d\} \Rightarrow \mathbf{card}(A) = 4$
\item $B = \{a, b, \{c\}, d, e, \{f, g, h, i\}\} \Rightarrow \mathbf{card}(B) = 6$
\item $C = \{1, 4, \{\emptyset\}, \{\emptyset, \{\emptyset\}\}\} \Rightarrow \mathbf{card}(C) = 4$
\end{itemize}

\subsection{Δυναμοσύνολο}
Το δυναμοσύνολο ενός συνόλου A ονομάζεται το σύνολο όλων των υποσυνόλων του Α συμπεριλαμβανομένου του κενού συνόλου και το ιδίου του συνόλου Α. Συμβολισμός: $\mathcal{P}(A)$

\end{document}